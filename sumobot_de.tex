\documentclass[a4paper,12pt]{article}
\usepackage[ngerman]{babel}
\usepackage{ucs}
\usepackage{multirow}
\usepackage{xltxtra}
\usepackage[utf8x]{inputenc}
\usepackage{fontspec}
\usepackage{eurosym}
\usepackage{graphicx}
\usepackage[paper=a4paper,left=25mm,right=25mm,top=25mm,bottom=25mm]{geometry}
\usepackage{makecell}
\usepackage[table]{xcolor}
\usepackage{float}
\usepackage[normalem]{ulem}
\usepackage{xcolor,colortbl}
\definecolor{Gray}{gray}{0.85}
\usepackage[automark]{scrlayer-scrpage}
\setlength{\parindent}{0em}
\setlength{\parskip}{1ex}
\pagestyle{scrheadings}
\clearscrheadfoot
\setmainfont[Mapping=tex-text]{Liberation Serif}
\begin{document}
\input{theme.tex}
\input{version.tex}
\ohead{Regelstand: \commitDate, id: \commitID}
\title{\tagYear\ SUMO Challenge Regeln}

\makeatletter
\let\inserttitle\@title
\makeatother
\begin{center}
	\rrgerLogo
	\huge                      % Schriftgröße einstellen
	\bfseries                   % Fettdruck einschalten
	\\
	\inserttitle
\end{center}
\section{Ziel}
Entwurf, Bau und Programmierung eines autonomen Roboters, der einen gegnerischen Sumoroboter suchen und aus einem erhöhten Ring schieben kann.
\section{Divisionen/Massenklassen}
Bitte entnehmen Sie der untenstehenden Tabelle, in welcher Division/Massenklasse Sie teilnehmen möchten.
\begin{center}
	\begin{tabular}{|c|c|c|} \hline
		\multirow{2}*{1kg} & \multirow{2}*{2kg} & Nur LEGO 1Kg \\
		& & (Optional) \\ \hline
		ES &  & ES \\ \hline
		MS & ** & MS \\ \hline
		HS & HS & HS \\ \hline
		** & UP & ** \\ \hline
	\end{tabular} \\ \vspace{\baselineskip}
\end{center}

\section{Roboter}
Autonomer Roboter, jede beliebige Plattform, der 1.500 USD oder weniger kostet und die folgenden Designbeschränkungen erfüllt, die beim Robot Check-In überprüft werden.
\begin{itemize}
	\item Die maximale Größe des Roboters beträgt 25 cm x 18 cm ohne Höhenbeschränkung, gemessen mit allen beweglichen Komponenten in ihrer Ausgangsposition.
	\item Größen- und Massenbeschränkungen werden während der gesamten Veranstaltung strikt durchgesetzt, um den Wettbewerb für alle Teilnehmer fair zu gestalten.
	\item Gelenkige oder bewegliche Komponenten sind zulässig, solange sie den oben genannten Konstruktionsregeln entsprechen. Es gilt jedoch die Regel "`kein vorsätzlicher Schaden"' - das bedeutet, dass Flossen und Gleitplatten zwar in Ordnung sind, aber absichtlich zerstörende Mechanismen wie Schleifspinner oder Hämmer etc. nicht erlaubt sind.
	\item In der Division "`Nur LEGO 1Kg"': Jeder LEGO-Roboter kann verwendet werden, muss aber vollständig aus LEGO-Markenteilen bestehen, autonom sein und den Designvorgaben entsprechen.
\end{itemize}

\section{Wettbewerbsring}
Ein schwarzer Kreis mit 100 cm Durchmesser und 5 cm weißem Rand auf 13 bis 20 mm dicker Platte. Die Oberfläche des Rings sollte 50 bis 80 mm über dem Boden liegen. (die Maße werden je nach den verwendeten lokalen Materialien leicht variieren).

\section{Allgemeine Spielregeln und Punktevergabe}
\begin{itemize}
	\item Jeder Roboter wird in einer Reihe von Wettkämpfen nach der "`Reihum-Methode"' antreten. Die Anzahl der Runden/Matches werden am Tag des Wettkampfes auf der Grundlage der Zeit und Anzahl der Roboter in jeder Kategorie bestimmt.
	\item Ein Spiel ist beendet, wenn eine Mannschaft zweimal gegen ihren Gegner gewonnen hat. 3 Punkte die für einen Sieg, 1 Punkt für ein Unentschieden und 0 Punkte für eine Niederlage vergeben werden.
	\item Die Punkte der Mannschaften werden während des Wettkampfs ausgezählt und angezeigt. Die besten 8 Mannschaften in jeder Kategorie werden für die Endspiele ausgewählt.
	\item Die Mannschaften können auf offenen Spielfeldern trainieren, wobei sie sich mit anderen trainierenden Teams abwechseln müssen.
	\item Die Roboter beginnen, indem sie die weiße Linie an den einander gegenüberliegenden Seiten des Tisches berühren. Sie können in jeder beliebigen Ausrichtung positioniert werden. Die Roboter müssen nach dem Drücken der Starttasten 3 Sekunden lang pausieren, damit das Teammitglied sich vom Ring zurückziehen kann.
	\item Der Verlierer ist der Roboter, der den Ring zuerst verlässt, was als Berührung der Oberfläche definiert ist, auf der der Wettbewerbsring platziert ist. Der Schiedsrichter kann nach 60 Sekunden ein Unentschieden ausrufen oder bei "`gesperrten Robotern"' nach 5 Sekunden einen Neustart erzwingen, wenn er dies für richtig hält.
	\item Die Roboterbediener dürfen ihre Roboter nicht berühren, es sei denn, der Schiedsrichter weist sie dazu an. Pro Spiel werden 5 Minuten zugeteilt, wenn es in dieser Zeit keinen Sieger gibt, wird es als unentschieden gewertet.
	\item Konfliktlösung - während des Spiels sind die Entscheidungen des Schiedsrichters endgültig.
\end{itemize}


\section{Turnierplan}
\begin{itemize}
	\item Wir veranstalten in der Regel Turniere für die besten 8 Mannschaften. Sollte es jedoch aufgrund von Gleichstand mehr als 8 Teams geben, kann der Veranstaltungsleiter die Turniergröße auf 12 oder 16 erhöhen oder ein Turnier mit Gleichstand veranstalten, um auf 8, 12 oder 16 Teams zu kommen, die ein Turnier ausrichten.
	\item Die aufsteigenden Teams werden entsprechend ihrer Gesamtpunktzahl in die Turnierklasse gesetzt. Unten sehen Sie ein Beispiel für unsere typische 8-Mannschafts-Turnierrunde.
	\item Der zweite Platz ("`Runner Up"') wird verwendet, um den 3. Platz auf der Grundlage des Ergebnisses der Halbfinalrunde zu bestimmen.
\end{itemize}
\begin{figure}[H]
	\centering
	\def\svgwidth{\columnwidth}
	\input{tournament_score/tournament_score.pdf_tex}
\end{figure}
\end{document}
